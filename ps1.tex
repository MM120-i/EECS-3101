\documentclass[12pt]{article}
\usepackage{geometry}
\geometry{letterpaper}
\usepackage{amssymb}
\usepackage{amsmath}
\usepackage{listings}
\usepackage{fancyhdr}
\usepackage{hyperref}
\usepackage{graphicx}
\usepackage{ulem}
\newcounter{ProblemNum}
\newcounter{SubProblemNum}[ProblemNum]
\newcommand{\divides}{\bigm|}
\renewcommand{\theProblemNum}{\arabic{ProblemNum}}
\renewcommand{\theSubProblemNum}{\alph{SubProblemNum}}
\newcommand*{\anyproblem}[1]{\newpage\subsection*{#1}}
\newcommand*{\problem}[1]{\stepcounter{ProblemNum} %
\anyproblem{Problem \theProblemNum. \; #1}}
\newcommand*{\soln}[1]{\subsubsection*{#1}}
\newcommand*{\solution}{\soln{Solution}}
\renewcommand*{\part}{\stepcounter{SubProblemNum} %
\soln{Part (\theSubProblemNum)}}


% Document metadata




% Document starts here
\begin{document}
\begin{center}
\begin{Large}
  \textbf{EECS3101Summer 2024 Assignment 1 }\\
\end{Large}
\begin{large}
	Due: Jun 17, 23:59
\end{large}
\end{center}
%\vspace{-.5in}

\section*{General Instructions}
Please read the following instructions carefully before starting the exercise. They contain important
information about general exercise expectations, exercise submission instructions,
and reminders of course policies.

\begin{itemize}
\item Your problem set is graded on both correctness and clarity of communication. Solutions
which are technically correct but poorly written will not receive full marks. Please read over
your solutions carefully before submitting them.

\item Each problem set must be completed individually

\item Solutions must be typeset electronically, submitted as a PDF with the correct filename ps1.pdf. Our recommendation goes for using {\LaTeX}  and we recommend \href{https://www.overleaf.com/}{Overleaf} as a tool, but you may feel free to pick your own tool, or generate a PDF using means such as Microsoft Word or other software.


\item Submissions must be made \emph{before} the due date and time on eclass. Late submissions are not accepted.

\end{itemize}


\problem{}
\textsc{(30 marks)}
The algorithm $reverse(lst)$ below reverses a list in place. More precisely, it satisfies the following precondition/postcondition
pair:
\vskip5pt
{\bf Precondition}: $lst$ is an array of any length
\vskip2pt
{\bf Postcondition}: lst upon completion of this algorithm, is mutated to the reversed list. 
For example, if the list prior reversal is [1, 2, 3], after the algorithm is completed, the content of the list is [3, 2, 1].

\begin{verbatim}
functon reverse(lst):
    n = len(lst)
    i, j = 0, n-1
    while i <= j:
       swap(lst[i], lst[j])
       i = i + 1
       j = j - 1
\end{verbatim}

Provide a complete proof of correctness of this algorithm performing all four steps.

%%Write your solution here

\problem{(30 marks)} Consider this alogrithm:

\begin{verbatim}
// pre: array A of length n, each A[i] is picked randomly uniformly 
// from the set {0,1,2,3,4,5,6,7,8,9}.
// post: return the index of the first occurrence of "check digit" of A 
// (already computed in the code)
// return -1 if check digit is not found
int FindCheckDigit(A) {
1   check_digit = 0
2   for (i=0; i < n; i++)
          check_digit = check_digit + A[i]
3   check_digit = check_digit % 10 # % means remainder of division
4   for (i = 0; i < n; i++) {
5         if (A[i] == check_digit) {
6                return i
7         }
8   }
9   return -1;
}
\end{verbatim}

Compute the average runtime for this algorithm. Show all details of your computation for both loops.

%%Write your solution here

\problem{(20 marks)} Consider this function:

\begin{verbatim}
boolean freaky(n) {
// PRE: n is a positive integer
// POST: ???
if n < 2:
     return false
x = 2
while x < n {
   if n mod x == 0
         return false
   x =  x + 1
}
return true
}
\end{verbatim}
\begin{enumerate}
\item[(i)] (5 marks)  Find the postcondition of this function
\item[(ii)] (5 marks ) Find the best possible big-oh bound for its complexity
\item[(iii)] (Just think ... no marks for this part). What can we say about big-theta complexity for this function?
\end{enumerate}

%%Write your solution here

\problem{(20 marks)}
\textsc{(10 marks)} \textbf{Big-Oh}. 
For each function $f$ in the left column of the following table, choose one expression $\mathcal{O}(g)$ from the following list: \\


$\mathcal{O}(\frac{1}{n})$,$\mathcal{O}(1)$,$\mathcal{O}(\log_2{n})$,$\mathcal{O}(n)$,$\mathcal{O}(n\log_2{n})$,$\mathcal{O}(n^2)$,$\mathcal{O}(n^{10})$,$\mathcal{O}(2^n)$,$\mathcal{O}(10^n)$,$\mathcal{O}(n^n)$\\


such that $f\in\mathcal{O}(g)$. Use each expression only once. To make it easy, you may simply copy and paste the tex code for each answer, in the correct table cell.

\vskip10pt

\begin{tabular}{|l|l|}
\hline
$f$&$\mathcal{O}(g)$\\
\hline
$3\cdot 2^n$&\\
\hline
$\frac{2n^4+1}{n^3+2n-1}$&\\
\hline
$(n^5+7)(n^5-7)$&\\
\hline
$\frac{n^4-n\log_2{n}}{n^2+1}$&\\
\hline
$\frac{n\log_2{n}}{n-5}$&\\
\hline
$8+\frac{1}{n^2}$&\\
\hline
$2^{3n+1}$&\\
\hline
$n!$&\\
\hline
$\frac{5\log_2{n+1}}{1+n\log_2{3n}}$&\\
\hline
$(n-1)\log_2(n^3+4)$&\\
\hline
\end{tabular}



%%Write your solution here



\end{document}