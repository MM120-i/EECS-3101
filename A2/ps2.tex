\documentclass[12pt]{article}
\usepackage{geometry}
\geometry{letterpaper}
\usepackage{amssymb}
\usepackage{amsmath}
\usepackage{listings}
\usepackage{fancyhdr}
\usepackage{hyperref}
\usepackage{graphicx}
\usepackage{ulem}
\newcounter{ProblemNum}
\newcounter{SubProblemNum}[ProblemNum]
\newcommand{\divides}{\bigm|}
\renewcommand{\theProblemNum}{\arabic{ProblemNum}}
\renewcommand{\theSubProblemNum}{\alph{SubProblemNum}}
\newcommand*{\anyproblem}[1]{\newpage\subsection*{#1}}
\newcommand*{\problem}[1]{\stepcounter{ProblemNum} %
\anyproblem{Problem \theProblemNum. \; #1}}
\newcommand*{\soln}[1]{\subsubsection*{#1}}
\newcommand*{\solution}{\soln{Solution}}
\renewcommand*{\part}{\stepcounter{SubProblemNum} %
\soln{Part (\theSubProblemNum)}}


% Document metadata




% Document starts here
\begin{document}
\begin{center}
\begin{Large}
  \textbf{EECS3101 Summer 2024 Assignment 2 }\\
\end{Large}
\begin{large}
	Due: Jul 10th, 2024
\end{large}
\end{center}
%\vspace{-.5in}

\section*{General Instructions}
Please read the following instructions carefully before starting the exercise. They contain important
information about general exercise expectations, exercise submission instructions,
and reminders of course policies.

\begin{itemize}
\item Your problem set is graded on both correctness and clarity of communication. Solutions
which are technically correct but poorly written will not receive full marks. Please read over
your solutions carefully before submitting them.

\item Each problem set must be completed individually

\item Solutions must be typeset electronically, submitted as a PDF with the correct filename ps2.pdf. Our recommendation goes for using {\LaTeX}  and we recommend \href{https://www.overleaf.com/}{Overleaf} as a tool, but you may feel free to pick your own tool, or generate a PDF using means such as Microsoft Word or other software.


\item Submissions must be made \emph{before} the due date and time on eclass. Late submissions are not accepted.

\end{itemize}



\problem{}
%%%%%%%%%%%%%%%
%%%%%%%%%%%%%%
\textsc{(30 Marks, 10 marks each part)} Recall exercise 4.4 from Week 5 lecture notes:

\vskip5pt

\includegraphics[scale=0.4]{ex4}

\vskip5pt

In what follows, we will call a \emph{matrix} any matrix that satisfies the conditions shown in the image above.

\vskip5pt

We will addittionally assume, that the matrix may have missing numerical elements. In this case, the cell fill be filled with the special value \texttt {INF} which is larger than any number (like infinity in mathematics). If a row contains an \texttt{INF} value this must be the last value, and if a column contains an \texttt{INF} value this must be the last value as well. For example, if a matrix contains a single \texttt{INF} value (and the rest is filled with numbers) the \texttt{INF} value must be located in the bottom row of the rightmost column. If a matrix contains no numbers (a.k.a \emph{empty matrix}), it will be filled only with \emph{INF} values.

\vskip5pt

In the form a warmup exercise, draw a $4\times 4$ matrix containing elements \\$\{9, 16, 3, 2, 4, 8, 5, 14, 12\}$.

\begin{enumerate}
	\item Develop an efficient algorithm called \texttt{popmin(A, n)} where $A$ is a nonempty $n\times n$ matrix which returns the smallest element of $A$ (which by definition is the element located in row 1, column 1) preserving the properties of the matrix as described above. Since we are removing a number from the matrix, the matrix after the operation will contain one more \texttt{INF} value. \textsc{Hint:} Think of \texttt{BuildMaxHeap} and don't forget to use recursion! The required operation must occur in place, that is no additional matrix (or additional array of any kind) can be used.
	
	\item Develop an eficient algorithm \texttt{insert(A, n, value)} where $A$ is a nonfull matrix (i.e. a matrix satisfying the above conditions, containin  at least one \texttt{INF} value), and \texttt{value} a number to be insterted. The insertion must occur in place, that is no additional matrix (or additional array of any kind) can be used.

	\item Using no other sorting method, but the algorithms developed in previous parts of this exercise (if needed), develop an algorithm that sorts $n^2$ numbers in $O(n^3)$ time. You must use a $n\times n$ matrix as desribed above.
\end{enumerate}



%%Write your solution here

\problem{}
\textsc{(20 marks)} 
Consider this algorithm:

\vskip5pt
\begin{verbatim}
// PRE: A an array of numbers, initially low = 0, high = size of A - 1
// POST: Returns the position of the pivot, and A is partitioned with 
//            elements to the left of the pivot being <= pivot, 
//            and elements to the right of the pivot being >=pivot
int partition(A, low, high):
    pivot = A[high]
    i = low-1
    for j from low to (high-1) do:
        if pivot >= A[j] then do:
            i = i + 1
            swap A[j] with A[i]
    swap A[i+1] with the pivot element at A[high]
    return (i+1)
\end{verbatim}


\begin{enumerate}
\item (\textsc{30 Points}) Prove correctenss of \texttt{partition}.
\item (\textsc{10 Points})  Use \texttt{partition} to write a version of \texttt{Quicksort} that sorts a numerical array A in place.
\end{enumerate}
%%Write your solution here

\problem{}
%%%%%%%%%%%%%%%
%%%%%%%%%%%%%%
\textsc{(30 Marks)} Consider the following three methods of solving a particular problem (input size $n$):
\begin{enumerate}
	\item
	You divide the problem into seven subproblems, each $\frac{1}{7}$ the size of the original problem, solve each recursively, then combine the results with a cost $12n+3$ where $n$ is the original problem size.
	
	\item
	You reduce the problem size by $2$,  solve the subproblem recursively, then combine the results with a cost 120 operations.
	
	\item
	You divide the problem into 2 subproblems, each $\frac{1}{2}$ of size of the original problem, solve each recursively, then combine the results with a cost $2n^2+3n+5$ where $n$ is the original problem size.
\end{enumerate}

Assume the base case has size 1 for all three methods.\\ \\
For each method, write a recurrence capturing its worst-case runtime.
Which of the three methods yields the fastest asymptotic runtime? 

In your solution, you should use the Master Theorem wherever possible. In the case where the Master Theorem doesn't apply, \emph{clearly state why not} based on your recurrence, and show your work solving the recurrence using another method (no proofs required).


%%Write your solution here

\problem{}
\textsc{(20 marks)} 

A number is called {\bf monotone} if it consists of repeated decimal digits. For example, 3333 and 7777 are monotone numbers. 

\begin{itemize}
\item[(i)] \textsc{(10 Marks)} Write a divide and conquer function (in pseudocode) with time complexity $\Theta(n)$ named \verb|multiply_monotone| that takes as input a two monotone numbers. We assume $n$ is maximum length (number of digits) of the two monotone numbers, and in turn, the length of each of them (for simplicity) may be taken to be a power of 2. You may assume a number is represented as a string of digits.
\item[(ii)] \textsc{(10 Marks)} Using Master Theorem, argue that the time complexity of your code is $\Theta(n)$.
\end{itemize}

%%Write your solution here
\end{document}
