\documentclass[12pt]{article}
\usepackage{geometry}
\geometry{letterpaper}
\usepackage{amssymb}
\usepackage{amsmath}
\usepackage{listings}
\usepackage{fancyhdr}
\usepackage{hyperref}
\usepackage{graphicx}
\usepackage{ulem}
\newcounter{ProblemNum}
\newcounter{SubProblemNum}[ProblemNum]
\newcommand{\divides}{\bigm|}
\renewcommand{\theProblemNum}{\arabic{ProblemNum}}
\renewcommand{\theSubProblemNum}{\alph{SubProblemNum}}
\newcommand*{\anyproblem}[1]{\newpage\subsection*{#1}}
\newcommand*{\problem}[1]{\stepcounter{ProblemNum} %
\anyproblem{Problem \theProblemNum. \; #1}}
\newcommand*{\soln}[1]{\subsubsection*{#1}}
\newcommand*{\solution}{\soln{Solution}}
\renewcommand*{\part}{\stepcounter{SubProblemNum} %
\soln{Part (\theSubProblemNum)}}


% Document metadata




% Document starts here
\begin{document}
\begin{center}
\begin{Large}
  \textbf{EECS3101 Summer 2024 Assignment 4 }\\
\end{Large}
\begin{large}
	Due: Aug 12th 23:59. \\ No extensions will be given, unless you are entitled to and extension by accessibility.
\end{large}
\end{center}
%\vspace{-.5in}

\section*{General Instructions}
Please read the following instructions carefully before starting the exercise. They contain important
information about general exercise expectations, exercise submission instructions,
and reminders of course policies.

\begin{itemize}
\item Your problem set is graded on both correctness and clarity of communication. Solutions
which are technically correct but poorly written will not receive full marks. Please read over
your solutions carefully before submitting them.

\item Each problem set must be completed individually

\item Solutions must be typeset electronically, submitted as a PDF with the correct filename ps4.pdf. Our recommendation goes for using {\LaTeX}  and we recommend \href{https://www.overleaf.com/}{Overleaf} as a tool, but you may feel free to pick your own tool, or generate a PDF using means such as Microsoft Word or other software.


\item Submissions must be made \emph{before} the due date and time on eclass. Late submissions are not accepted.

\end{itemize}



\problem{}
%%%%%%%%%%%%%%%
%%%%%%%%%%%%%%
\textsc{(40 Marks)} We are given a list $L$ of $n$ ordered integers you're tasked with removing $m$ of them such that the distance between the closest two remaining integers is maximized.  As an example, consider the list $[1,4,5,6,8,9]$, where we are allowed to remove two numbers. Here, an optimal solution would be to remove the numbers  5 and 8, leaving us with the list $[1,4,6,9]$. The distance between the closest remaining numbers is 2 (between 4 and 6). As it turns out there is no (known) greedy algorithm to solve this problem. However, there is a dynamic programming solution. Devise a dynamic programming solution which determines the maximum distance between the closest two points after removing $m$ numbers. Note, it doesn't need to return the resulting list itself.

\textbf{Hint 1:} Your sub-problems should be of the form $S(i,j)$, where $S(i,j)$ returns the maximum distance of the closest two numbers when only considering removing $j$ of the first $i$ numbers in $L$. As an example if $L = [3,4,6,8,9,12,13,15]$, then $S(4,1)=2$, since the closest two values of $L' = [3,4,6,8]$ are 6 and 8 after removing 4 (note, 8-6 = 2). 

\textbf{Hint 2:} For the sub-problem $S(i,j)$, assuming $j < i -2$, you know there's always an optimal solution which leaves the values $L[1]$ and $L[i]$ in the list.

Give Bellman equation for $S(i,j)$ including all its base cases.


%%Write your solution here

\problem{}
\textsc{(60 marks)} You're an entry level employee managing a financial portfolio. Because you're a total novice, your boss limits the portfolio to a single stock, and moreover, each day you're only allowed to make one of three choices: do nothing, purchase exactly one share, or sell all shares you currently own. You are scheduled to manage the portfolio for $n$ days, and for each day $i$ the price of the stock is estimated to be $p_i$. Assume these estimations are perfect. On any given day $i$ the value of your portfolio equals
$$ (\#\text{shares you own})p_i - (\text{amount spent to buy  shares}) + (\text{amount made from selling past shares}).$$ 
\begin{enumerate}
\item Devise a dynamic programming algorithm to determine the maximum value your portfolio can achieve on day $n$. Make sure to:
\begin{itemize}
\item Clearly state your sub-problems in plain English
\item Formally define your sub-problems mathematically (give Bellman equations)
\item State how one would arrive at the final answer from your sub-problems (where do you read the optimal solution on the table)
\end{itemize}
\end{enumerate}
%%Write your solution here
\end{document}
